% Options for packages loaded elsewhere
\PassOptionsToPackage{unicode}{hyperref}
\PassOptionsToPackage{hyphens}{url}
%
\documentclass[
]{article}
\usepackage{amsmath,amssymb}
\usepackage{lmodern}
\usepackage{ifxetex,ifluatex}
\ifnum 0\ifxetex 1\fi\ifluatex 1\fi=0 % if pdftex
  \usepackage[T1]{fontenc}
  \usepackage[utf8]{inputenc}
  \usepackage{textcomp} % provide euro and other symbols
\else % if luatex or xetex
  \usepackage{unicode-math}
  \defaultfontfeatures{Scale=MatchLowercase}
  \defaultfontfeatures[\rmfamily]{Ligatures=TeX,Scale=1}
\fi
% Use upquote if available, for straight quotes in verbatim environments
\IfFileExists{upquote.sty}{\usepackage{upquote}}{}
\IfFileExists{microtype.sty}{% use microtype if available
  \usepackage[]{microtype}
  \UseMicrotypeSet[protrusion]{basicmath} % disable protrusion for tt fonts
}{}
\makeatletter
\@ifundefined{KOMAClassName}{% if non-KOMA class
  \IfFileExists{parskip.sty}{%
    \usepackage{parskip}
  }{% else
    \setlength{\parindent}{0pt}
    \setlength{\parskip}{6pt plus 2pt minus 1pt}}
}{% if KOMA class
  \KOMAoptions{parskip=half}}
\makeatother
\usepackage{xcolor}
\IfFileExists{xurl.sty}{\usepackage{xurl}}{} % add URL line breaks if available
\IfFileExists{bookmark.sty}{\usepackage{bookmark}}{\usepackage{hyperref}}
\hypersetup{
  pdftitle={Simple markdown exercise},
  pdfauthor={Miguel Portela},
  hidelinks,
  pdfcreator={LaTeX via pandoc}}
\urlstyle{same} % disable monospaced font for URLs
\usepackage[margin=1in]{geometry}
\usepackage{longtable,booktabs,array}
\usepackage{calc} % for calculating minipage widths
% Correct order of tables after \paragraph or \subparagraph
\usepackage{etoolbox}
\makeatletter
\patchcmd\longtable{\par}{\if@noskipsec\mbox{}\fi\par}{}{}
\makeatother
% Allow footnotes in longtable head/foot
\IfFileExists{footnotehyper.sty}{\usepackage{footnotehyper}}{\usepackage{footnote}}
\makesavenoteenv{longtable}
\usepackage{graphicx}
\makeatletter
\def\maxwidth{\ifdim\Gin@nat@width>\linewidth\linewidth\else\Gin@nat@width\fi}
\def\maxheight{\ifdim\Gin@nat@height>\textheight\textheight\else\Gin@nat@height\fi}
\makeatother
% Scale images if necessary, so that they will not overflow the page
% margins by default, and it is still possible to overwrite the defaults
% using explicit options in \includegraphics[width, height, ...]{}
\setkeys{Gin}{width=\maxwidth,height=\maxheight,keepaspectratio}
% Set default figure placement to htbp
\makeatletter
\def\fps@figure{htbp}
\makeatother
\setlength{\emergencystretch}{3em} % prevent overfull lines
\providecommand{\tightlist}{%
  \setlength{\itemsep}{0pt}\setlength{\parskip}{0pt}}
\setcounter{secnumdepth}{-\maxdimen} % remove section numbering
\usepackage{hyperref}
\hypersetup{ colorlinks=true, linkcolor=blue, urlcolor=blue, }
\ifluatex
  \usepackage{selnolig}  % disable illegal ligatures
\fi
\newlength{\cslhangindent}
\setlength{\cslhangindent}{1.5em}
\newlength{\csllabelwidth}
\setlength{\csllabelwidth}{3em}
\newenvironment{CSLReferences}[2] % #1 hanging-ident, #2 entry spacing
 {% don't indent paragraphs
  \setlength{\parindent}{0pt}
  % turn on hanging indent if param 1 is 1
  \ifodd #1 \everypar{\setlength{\hangindent}{\cslhangindent}}\ignorespaces\fi
  % set entry spacing
  \ifnum #2 > 0
  \setlength{\parskip}{#2\baselineskip}
  \fi
 }%
 {}
\usepackage{calc}
\newcommand{\CSLBlock}[1]{#1\hfill\break}
\newcommand{\CSLLeftMargin}[1]{\parbox[t]{\csllabelwidth}{#1}}
\newcommand{\CSLRightInline}[1]{\parbox[t]{\linewidth - \csllabelwidth}{#1}\break}
\newcommand{\CSLIndent}[1]{\hspace{\cslhangindent}#1}

\title{Simple markdown exercise}
\author{Miguel Portela}
\date{2021}

\begin{document}
\maketitle

Good afternoon!

You can type in \textbf{bold} or \emph{italic}

You can put an \textbf{entire sentence in bold} \emph{or italic}

If you want to type a Header

\hypertarget{first-header}{%
\section{First header}\label{first-header}}

or

\hypertarget{insert-a-sub-section}{%
\subsection{Insert a sub-section}\label{insert-a-sub-section}}

\hypertarget{sub-sub-section}{%
\subsubsection{Sub sub-section}\label{sub-sub-section}}

\hypertarget{insert}{%
\section{Insert}\label{insert}}

\hypertarget{your-first-link}{%
\subsection{\texorpdfstring{Your first
\emph{link}}{Your first link}}\label{your-first-link}}

\href{https://www.markdowntutorial.com/}{This is the text you will see}

or \href{https://www.markdowntutorial.com/}{\textbf{using bold}}

You can also use reference links to move to
\href{http://www1.eeg.uminho.pt/economia/mangelo/}{\emph{\textbf{my
website}}} or to \href{https://www.uminho.pt/EN/}{\emph{Universidade do
Minho}}

\hypertarget{how-to-use-images-and-define-size}{%
\section{How to use images and define
size}\label{how-to-use-images-and-define-size}}

\hypertarget{first-using-a-relative-size}{%
\subsection{First using a relative
size}\label{first-using-a-relative-size}}

\includegraphics[width=0.31\textwidth,height=\textheight]{./images/uminho.png}

or a \textbf{fixed size}

\begin{figure}
\centering
\includegraphics[width=1.7in,height=0.7in]{./images/uminho.png}
\caption{Universidade do Minho}
\end{figure}

You can also use an image from a link:

\begin{figure}
\centering
\includegraphics[width=0.25\textwidth,height=\textheight]{https://rstudio.com/wp-content/uploads/2018/10/RStudio-Logo-Flat.png}
\caption{RStudio's logo}
\end{figure}

When writing your document you can quote using carat (\textgreater).
This is called a block quote. The following text was generated using
\href{https://jaspervdj.be/lorem-markdownum/}{Lorem Ipsum}.

\begin{quote}
Tuas vocat velantibus rogos, quem tamen foedere \textbf{laetabere
bipennifer} nulla se \textbf{camini}. In \textbf{forsitan}, in
\textbf{verti iam} submissaeque faciam adversum et quae. Tu ille
postquam interdum, est sed Britannos, dedecus solae pectore at modo in
adeste vitta tantummodo ingrato. Caedis numina, tonitruque iugum speciem
corpore, leves hunc Zancle auferat umero foribusque cursus negate
inhaerentem Styga Hebre meosque?

Spes speciosoque dixit ferinae agros simulatque domum alimentaque pabula
claro; a costis, a nec captus Aquilone. Pharonque donec, modo suo vires
arcanis, quem illis Vesta quae dedit. Pervenerat placat lenta; sine
finxit, tantum pater, tamen fila sedibus, sonent. Nempe caedit quas
fundunt optima sua vultusque remolliat et habet tendebat Hesperidas:
corniger. \emph{Digna} spatium effugere magne, a pectora hospes volant
frena crinis resonant protinus; morte Hippomenen.
\end{quote}

Another markdown element you may want to use are \textbf{lists}.

\hypertarget{course-outline}{%
\section{Course Outline}\label{course-outline}}

as an \emph{ordered list}

\begin{enumerate}
\def\labelenumi{\arabic{enumi}.}
\tightlist
\item
  Markdown and Pandoc
\item
  Create a markdown document and run code
\item
  Develop a report
\item
  Publish the report
\end{enumerate}

or just as bullets

\begin{itemize}
\tightlist
\item
  Markdown and Pandoc
\item
  Create a markdown document and run code
\end{itemize}

where you may need additional depth. Use tabulations

\begin{itemize}
\tightlist
\item
  Markdown and Pandoc

  \begin{itemize}
  \tightlist
  \item
    Create a markdown document and run code
  \end{itemize}
\item
  Develop a report

  \begin{itemize}
  \tightlist
  \item
    Publish the report
  \end{itemize}
\end{itemize}

A more complex example

\hypertarget{literate-programming-in-r-markdown}{%
\section{LITERATE PROGRAMMING IN R
MARKDOWN}\label{literate-programming-in-r-markdown}}

\begin{itemize}
\item
  Date: 27 \& 29 October \textbar{} 17h00-20h00
\item
  Delivered by: Miguel Portela, University of Minho

  Literate programming refers to melding a descriptive narrative and
  computer code into a single document, from which both human-friendly
  documentation and computer readable files can be created. Your work
  should be transparent, easy to update, easy to maintain, and easy to
  replicate. Literate programming saves time and effort, so we can
  dedicate more time doing research. Literate programming is also useful
  for teaching.
\end{itemize}

\textbf{Course Outline}

\begin{verbatim}
1.  Markdown and Pandoc
2.  Create a markdown document and run code
3.  Develop a report
4.  Publish the report
\end{verbatim}

\textbf{References}

\begin{itemize}
\item
  Xie, Y., Allaire, J.J. and Grolemund, G., 2018. R markdown: The
  definitive guide. CRC Press.
  (\url{https://bookdown.org/yihui/rmarkdown/}), ``\emph{Course 5:
  Web-based tools for data analysis: JupyterLab environment and workflow
  optimization}''
\item
  The Jupyter Notebook: \url{https://jupyter-notebook.readthedocs.io/}
\item
  \href{https://jupyter.org/}{Project Jupyter}
\end{itemize}

\hypertarget{paragraphs-tables}{%
\section{Paragraphs \& Tables}\label{paragraphs-tables}}

\hypertarget{using-a-double-space-at-the-end-of-the-sentence}{%
\subsection{Using a double space at the end of the
sentence}\label{using-a-double-space-at-the-end-of-the-sentence}}

Phaethon Delphos mea gravis excipiunt stabat: quem aqua taceam Phoebo,
vir aratri, Ulixes haec perque.\\
Nactus dempserat sui regnat enim, acta stet Areos praesagaque in iacent?
Fuerant crescentem vinci clamat.

\hypertarget{add-footnotes}{%
\subsection{Add footnotes}\label{add-footnotes}}

This is a footnote.\footnote{Ad remorum vestem pater victor Megareus
  lacrimas adsiduae regina sequenti Invidiae, ille tum aliquid. Locus
  uno quid curruque dixit, me regis, deum \textbf{iamque}, et ripas
  validum ubi! Auras amores quam feritatis apros demite ademptas est
  \textbf{tanto}!}

\hypertarget{this-is-a-table}{%
\subsection{This is a table}\label{this-is-a-table}}

\begin{longtable}[]{@{}lcr@{}}
\caption{Sample table}\tabularnewline
\toprule
Tables & Are & Cool \\
\midrule
\endfirsthead
\toprule
Tables & Are & Cool \\
\midrule
\endhead
Var 1 is & Left-Aligned & \$1271 \\
Var 2 is & Centered & \$13 \\
Var 3 is & Right-Aligned & \$7 \\
\bottomrule
\end{longtable}

\hypertarget{the-yaml-concept}{%
\section{The YAML concept}\label{the-yaml-concept}}

You can add information to your document, like title, author, etc.,
using \href{https://yaml.org/}{YAML}.

\begin{quote}
YAML: YAML Ain't Markup Language
\end{quote}

\begin{quote}
What It Is: YAML is a human friendly data serialization standard for all
programming languages.
\end{quote}

The following lines are comments so \emph{pandoc} will not compile them.
You can use standard HTML tags to comments out sections of your code. To
see their purpose add them to the beginning of the text.

\hypertarget{citations}{%
\subsection{Citations}\label{citations}}

Lorem markdownum medulla: Est hanc instrumenta sibi; premit opem Dianae,
\emph{ubi India} vocesque prodamne, quamvis? Et esse. Quod molire
auxiliumque caelumque tertia hospes, fecerat sermonibus prensamque
mortale summa, iubeatis coercet iugulum, \textbf{et}. For further
discussion (see Solow, 1952:pp.31--32).

\hypertarget{equations}{%
\section{Equations}\label{equations}}

You can write inline equations as \(y_i = \alpha_0 + \tau x_i + \psi_i\)
or numbered equations,

\begin{equation}
y_{it} = \beta_0 + \beta_1 x_{it} + \eta_i + \varepsilon{it}
\end{equation}

\hypertarget{final-remarks}{%
\section{Final remarks}\label{final-remarks}}

For additional insights see MacFarlane (2020).

\hypertarget{references}{%
\section*{References}\label{references}}
\addcontentsline{toc}{section}{References}

\hypertarget{refs}{}
\begin{CSLReferences}{1}{0}
\leavevmode\hypertarget{ref-MacFarlane}{}%
MacFarlane, J. (2020) {Pandoc User's Guide}. \emph{Link:
https://pandoc.org/MANUAL.pdf}.

\leavevmode\hypertarget{ref-solow1952structure}{}%
Solow, R. (1952) On the structure of linear models. \emph{Econometrica}.
20 (1), 29--46.

\end{CSLReferences}

\end{document}
